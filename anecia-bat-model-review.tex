% Options for packages loaded elsewhere
\PassOptionsToPackage{unicode=true}{hyperref}
\PassOptionsToPackage{hyphens}{url}
%
\documentclass[
]{article}
\usepackage{lmodern}
\usepackage{amssymb,amsmath}
\usepackage{ifxetex,ifluatex}
\ifnum 0\ifxetex 1\fi\ifluatex 1\fi=0 % if pdftex
  \usepackage[T1]{fontenc}
  \usepackage[utf8]{inputenc}
  \usepackage{textcomp} % provides euro and other symbols
\else % if luatex or xelatex
  \usepackage{unicode-math}
  \defaultfontfeatures{Scale=MatchLowercase}
  \defaultfontfeatures[\rmfamily]{Ligatures=TeX,Scale=1}
\fi
% Use upquote if available, for straight quotes in verbatim environments
\IfFileExists{upquote.sty}{\usepackage{upquote}}{}
\IfFileExists{microtype.sty}{% use microtype if available
  \usepackage[]{microtype}
  \UseMicrotypeSet[protrusion]{basicmath} % disable protrusion for tt fonts
}{}
\makeatletter
\@ifundefined{KOMAClassName}{% if non-KOMA class
  \IfFileExists{parskip.sty}{%
    \usepackage{parskip}
  }{% else
    \setlength{\parindent}{0pt}
    \setlength{\parskip}{6pt plus 2pt minus 1pt}}
}{% if KOMA class
  \KOMAoptions{parskip=half}}
\makeatother
\usepackage{xcolor}
\IfFileExists{xurl.sty}{\usepackage{xurl}}{} % add URL line breaks if available
\IfFileExists{bookmark.sty}{\usepackage{bookmark}}{\usepackage{hyperref}}
\hypersetup{
  pdftitle={A review of mechanistic models of viral dynamics in bat reservoirs for zoonotic disease},
  pdfauthor={Anecia Gentles},
  hidelinks,
}
\urlstyle{same} % disable monospaced font for URLs
\usepackage[margin=1in]{geometry}
\usepackage{graphicx,grffile}
\makeatletter
\def\maxwidth{\ifdim\Gin@nat@width>\linewidth\linewidth\else\Gin@nat@width\fi}
\def\maxheight{\ifdim\Gin@nat@height>\textheight\textheight\else\Gin@nat@height\fi}
\makeatother
% Scale images if necessary, so that they will not overflow the page
% margins by default, and it is still possible to overwrite the defaults
% using explicit options in \includegraphics[width, height, ...]{}
\setkeys{Gin}{width=\maxwidth,height=\maxheight,keepaspectratio}
\setlength{\emergencystretch}{3em} % prevent overfull lines
\providecommand{\tightlist}{%
  \setlength{\itemsep}{0pt}\setlength{\parskip}{0pt}}
\setcounter{secnumdepth}{-\maxdimen} % remove section numbering
% Redefines (sub)paragraphs to behave more like sections
\ifx\paragraph\undefined\else
  \let\oldparagraph\paragraph
  \renewcommand{\paragraph}[1]{\oldparagraph{#1}\mbox{}}
\fi
\ifx\subparagraph\undefined\else
  \let\oldsubparagraph\subparagraph
  \renewcommand{\subparagraph}[1]{\oldsubparagraph{#1}\mbox{}}
\fi

% Set default figure placement to htbp
\makeatletter
\def\fps@figure{htbp}
\makeatother


\title{A review of mechanistic models of viral dynamics in bat reservoirs for
zoonotic disease}
\author{Anecia Gentles}
\date{}

\begin{document}
\maketitle

If \href{https://doi.org/10.1080/20477724.2020.1833161}{my new paper}
made you feel like this\ldots{}

Here's a little explanation.

\begin{center}\rule{0.5\linewidth}{0.5pt}\end{center}

\textbf{On mechanistic models\ldots{}}

The magnitude of the SARS-CoV-2 (or COVID-19) pandemic has brought the
work of medical professionals, epidemiologists and disease ecologists to
the forefront of the world's collective consciousness; and, for the
first time in a while, everyone is asking how exactly did a bat virus
become a very big human problem. As we learn more and more about how
this virus spreads, epidemiologists continue to improve models of human
to human transmission by building upon a framework called a
compartmental or mechanistic model. Known as the SIR (Susceptible,
Infected, Recovered) model, this mechanistic model of transmission
tracks the progression of a pathogen within an individual (cell to
cell), a population (host to host), or across metapopulations by
categorizing each entity by their infection status at a given time.
These statuses include, but aren't limited to: Susceptible, Infected,
and Recovered; and the sum of the individuals of each status is equal to
N, the population. A system of calculus equations is used to describe an
individual's change in status over time. The equations are modified by
parameters such as the transmission coefficient (β), the rate of
recovery (λ), and processes such as the death (θ) and birth (Ω) rate.

\begin{center}\rule{0.5\linewidth}{0.5pt}\end{center}

\textbf{On mechanistic models of bat-virus systems}

Disease ecologists also use the SIR model to understand transmission
dynamics in animal species. SARS-CoV- 2 is believed to have originated
from a Betacoronavirus lineage circulating in Rhinolophus spp. horseshoe
bats of South-Central China. Although bats are the reservoir hosts of an
impressive number of potentially zoonotic viruses, very little is known
about the transmission dynamics or mechanisms that allow virus to be
maintained in bat populations. This is due in part to the difficulty of
catching bats, detecting active infections, and determining age for
long-term studies. In this paper, we've found that there are only 25
mechanistic transmission models of bat-borne viruses published in the
scientific literature. Of these models, thirteen are theoretical -- in
which the parameters are ``borrowed'' from similar systems, and twelve
are fitted to data, with parameters estimated from data collected in a
field system. We focused the majority of our analysis on the data-fitted
models.

\begin{center}\rule{0.5\linewidth}{0.5pt}\end{center}

\textbf{Lyssaviruses}

The available literature focuses on three lyssaviruses in several bat
species: rabies virus (RABV) in North American big brown bats (Eptesicus
fuscus) and Peruvian vampire bats (Desmodus rotundus); European bat
lyssavirus type 1 (EBLV-1) in four insectivorous bat species in Spain;
and Lagos Bat Virus (LBV) in African straw-colored fruit bats (Eidolon
helvum). Broadly, bat-lyssaviruses can be represented by an `SEIR' model
framework -- in which `E' stands for the `Exposed' class. Generally, the
RABV studies that we reviewed agree that bats will either die or
seroconvert (test positive for RABV antibodies) after exposure to an
infected individual, and repeated exposure may lead to immunity. These
variations in response to exposure may play a large part in maintaining
rabies virus in bat populations. Bats exposed or infected with EBLV-1 or
LBV also show similar dynamics, though migratory behavior and
co-roosting amongst the Spaniard bat species appears to play a role in
maintaining EBLV-1, and African straw-colored fruit bats may become
immune for life after surviving an LBV infection.

\begin{center}\rule{0.5\linewidth}{0.5pt}\end{center}

\textbf{Filoviruses}

Filoviruses infamously include the genera Marburgvirus and Ebolavirus,
both of which are believed to be maintained in bat reservoirs. Of the
seven mechanistic models of bat-filovirus systems that we reviewed, only
one (by Ekipa Fanihy!) was fitted to data. This study fitted an `MSIRN'
model to age-structured serological data collected from Madagascan
flying foxes (Pteropus rufus). Here, the M stands for
`Maternally-derived immunity' -- a process by which newborn bats receive
antibodies from their mothers, and `N' (non-antibody mediated immune)
represents a class of individuals who maintained lifelong immunity not
detectable by an antibody response. The model proposes an explanation
for the age-structured serological data, hypothesizing that
cell-mediated or innate immunity may play a role in bat filovirus
maintenance. The low force of infection (or per capita rate at which
individuals become infected) estimated for this flying fox population
suggests that spatial metapopulation dynamics, in which infections blink
in and out of local populations, may play a role in virus maintenance in
this system.

\begin{center}\rule{0.5\linewidth}{0.5pt}\end{center}

\textbf{Henipaviruses}

Bats infected with henipaviruses shed virus in their urine. This allows
researchers to noninvasively collect urine samples beneath roosts. Two
of the three focal studies of this section modeled transmission of
Ghanaian henipavirus (GhV) collected from E. helvum bats sampled from
populations across Africa, and from a captive colony in Ghana. The third
study, by Ekipa Fanihy, modeled an undescribed henipavirus identified
serologically in the Madagascan fruit bat, Eidolon dupreanum. The
pan-African and Malagasy studies came to similar conclusions, showing an
important role for waning immunity in explaining patterns in the
henipavirus data, though MSIRS model offered the best match to the GhV
data while another MSIRN model recovered the Malagasy henipavirus data.

\begin{center}\rule{0.5\linewidth}{0.5pt}\end{center}

\textbf{Coronaviruses}

At the time of the writing of our paper, we found only one data-fitted
model of a coronavirus system! The authors of this study found that an
SIRS model that allows some bats to maintain a long (`persistent')
infectious period best fit the transmission dynamics of an undescribed
alphacoronavirus in the large-footed myotis bats (Myotis macropus) of
Australia. Given our current situation, many more models of coronavirus
dynamics in bat systems are likely to emerge in the upcoming years!

\end{document}
